Antrenarea modelelor de învățare automată de tip "deep learning" necesită cantități mari de date de înaltă calitate.
Cu toate acestea, obținerea de adnotări pentru
date reale are un cost ridicat și prezintă probleme legate de diversitatea datelor și de licențiere.
Generarea de date sintetice evită unele dintre aceste limitări, 
dar diversitatea datelor rămâne nesatisfăcătoare atât timp cât
sunt utilizate modele 3D statice. Pentru a rezolva această problemă, prezentăm The Procedural
Outdoors Scene Generator, un proiect Python care combină date din diverse 
Sisteme informaționale Geografice cu tehnici de modelare procedurală, pentru a genera scene realiste cu o înaltă
diversitate și variație a datelor. Demonstrăm eficacitatea sistemului prin generarea unor seturi de date care
să fie utilizate pentru antrenarea unui model de învățare automată care detectează șine de cale ferată.